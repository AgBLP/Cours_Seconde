%%%% french character
\usepackage[french]{babel}
\usepackage[T1]{fontenc}
\usepackage[utf8]{inputenc}

%%%% useful packages
\usepackage[a4paper, left=1.3cm, right=1.3cm, top=2.2cm, bottom=2.3cm]{geometry}
\usepackage{subcaption} % for figure caption
\usepackage{graphicx} % image
\usepackage{tabularx} % table
\usepackage[table]{xcolor} % color in table
\usepackage{amsmath} % math
\usepackage{amssymb} % bold math
\usepackage{wasysym} % integral
\usepackage[many]{tcolorbox} % colored box
\usepackage{awesomebox} % pour des box déjà définies
\usepackage{fancyhdr} % headers
\usepackage{enumitem} % for bullet in itemize
\usepackage[colorlinks=true,linkcolor=black,citecolor=black,filecolor=black,urlcolor=black]{hyperref} % for link
\usepackage{accents} % for complex notation
\usepackage[european, straightvoltages, RPvoltages]{circuitikz} % for electronic circuit

\usepackage{hyperref}   %pour les liens url et les références
\hypersetup{
    colorlinks=true,
    urlcolor=blue,
    linkcolor=blue,
    breaklinks=true
}
\usepackage{empheq}
\usepackage{multicol} % to use several columns
%\setlength{\columnseprule}{1pt} %Separator ruler width
\usepackage{cellspace}  % espace du texte dans les colonnes tableaux
\usepackage{colortbl}  %pour colorier les cases de tableaux
\tcbuselibrary{skins} %library de 
\usepackage{shadow}
\usepackage{pifont}

\usepackage{fontawesome5}

%\usepackage{fontawesome} % awesome icons
\usepackage{ifthen} % for loop and boolean in commands
\usepackage{qrcode}
\usepackage{pdfpages} % to include pdf
\usepackage{wrapfig} % to wrap text around figures
\usepackage{chemfig} % to draw chemistry formula
\usepackage{chemist} % surtout pour chemform
\usepackage{multirow} % for vertically merged cells
\usepackage{makecell} % to format cell in tables
\usepackage{physics} % for derivatives, braket, etc.
\usepackage{esvect} % for large vectors
\usepackage{listings} % for code
\usepackage{dashundergaps} % for automatic text to fill
% dyslexia friendly font (need to be compiled in xetex)
%\usepackage{fontspec}
%\setmainfont{OpenDyslexic}
\usepackage{tikz}
\usepackage{pgfplots}
\usepackage[framemethod=tikz]{mdframed} %pour les boîtes
\usepackage{dashundergaps} %pour les textes à trous

%%%% settings
\setlength{\parskip}{0cm}
\setlength{\parindent}{0cm}
\renewcommand{\baselinestretch}{1.3}
\setcounter{tocdepth}{2}
\renewcommand{\thesection}{\textcolor{red}{\Roman{section}}}
\renewcommand{\thesubsection}{\textcolor{red}{\Roman{section}.\arabic{subsection}}}

%%%% tikz configuration
\usetikzlibrary{babel}
\tikzset{>=latex}
\usetikzlibrary{shadows}
\usetikzlibrary{backgrounds}



%%%% header
\renewcommand{\headrulewidth}{0.4pt}
\setlength{\headheight}{22.50113pt}


%%%% Table
\renewcommand{\tabularxcolumn}[1]{m{#1}}
\setlength{\extrarowheight}{8pt}
\newcolumntype{P}[1]{>{\centering\arraybackslash}p{#1\linewidth}}% colonne de type p mais centrée
\cellspacetoplimit 2pt %espace au dessus du texte
\cellspacebottomlimit 2pt  %espace en dessous du texte



%%%% Chemfig configuration
\setchemfig{
  atom sep=20pt,
  bond style={line width=1pt},
  angle increment=30
}


%%%% dashundergaps configuration
\dashundergapssetup{
  gap-numbers = false,
  gap-format = dot,
  gap-widen,
  gap-extend-percent
}