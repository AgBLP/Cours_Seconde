\modeCorrection

\nomPrenomClasse

\begin{center}
\begin{Large}
    
    Interrogation de cours : Corps purs et mélanges au quotidien (10min)
\end{Large}
\end{center}
\vspace{1cm}


\question{Définir un corps pur. Donner un exemple.}{un corps pur est constitué d'une unique espèce chimique. On peut prendre l'exemple de l'eau pure ou du graphite.}{2}

\question{Dans un récipient, on verse de l'huile et de l'eau. Qualifier le mélange obtenu. Comment peut-on qualifier les deux liquides ?}{Le mélange obtenu est hétérogène. Les deux liquides ne sont pas miscibles.}{2}

\question{Donner la formule de la masse volumique. Donner la signification et l'unité de chaque grandeur physique.}{La masse volumique est donnée par la formule :
\begin{equation*}
    \rho = \frac{m}{V}
\end{equation*} avec $\rho$ la masse volumique (en g.cm$^{-3}$ ou g.L$^{-1}$), avec $m$ la masse (en g ou kg) et V le volume de l'espèce chimique (en cm$^3$ ou m$^3$ ou L ou mL).}{3}
\\
\newline
\newline

\setcounter{exercice}{0}
\nomPrenomClasse

\begin{center}
\begin{Large}
    
    Interrogation de cours : Corps purs et mélanges au quotidien (10min)
\end{Large}
\end{center}
\vspace{1cm}


\question{Définir un mélange. Donner un exemple.}{Un mélange est constitué de plusieurs espèces chimiques. On peut prendre l'exemple de l'huile ou du miel.}{2}

\question{Dans un récipient, on verse de l'huile et beaucoup de poivre. Qualifier le mélange obtenu. Comment qualifier deux liquides formant un mélange homogène ?}{Le mélange obtenu est hétérogène. Les deux liquides sont miscibles.}{2}

\question{Donner la formule de la masse volumique. Donner la signification et l'unité de chaque grandeur physique.}{La masse volumique est donnée par la formule :
\begin{equation*}
    \rho = \frac{m}{V}
\end{equation*} avec $\rho$ la masse volumique (en g.cm$^{-3}$ ou g.L$^{-1}$), $m$ la masse (en g ou kg) et V le volume de l'espèce chimique (en cm$^3$ ou m$^3$ ou L ou mL).}{3}
