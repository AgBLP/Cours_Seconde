%\modeCorrection

%%%% début de la page
\renewcommand{\thesection}{\textcolor{red}{Partie \Roman{section} -}}
\renewcommand{\thesubsection}{\textcolor{red}{\Roman{section}.\arabic{subsection}}}
\renewcommand{\thesubsubsection}{\textcolor{red}{\Roman{section}.\arabic{subsection}.\alph{subsubsection}}}

\setcounter{section}{0}
\setcounter{document}{0}
\sndEnTeteTPNeuf

\begin{center}
\begin{mdframed}[style=titr, leftmargin=60pt, rightmargin=60pt, innertopmargin=7pt, innerbottommargin=7pt, innerrightmargin=8pt, innerleftmargin=8pt]

\begin{center}
\large{\textbf{TP 9 : Phénomènes de réflexion en réfraction de la lumière
}}
\end{center}
\end{mdframed}
\end{center}

%\begin{tableauCompetences}
%    APP & Exploiter des explications orales pour rédiger un protocole & & & & \\
 %   \hline
  %  REA & Réaliser une série de mesures ; relever les résultats obtenus & & & & \\
   %  \hline 
    % REA & Utiliser une grandeur quotient pour déterminer le numérateur ou le dénominateur& & & & \\
     %\hline 
   % COM & Rendre compte de façon écrite & & & & \\
    %\hline
    %VAL & Analyser l’ensemble des résultats de façon critique  & & & &
%\end{tableauCompetences}


%%%% objectifs
\begin{tcolorbox}[colback=blue!5!white,colframe=blue!75!black,title=Objectifs de la séance :]
\begin{itemize}
    \item Mettre en évidence les phénomène de réflexion et de réfraction de la lumière ;
    \item Identifier une relation de proportionnalité ;
    \item Valider une loi en réalisant une série de mesures 
    \item Déterminer l'indice de réfraction d'un milieu ;
\end{itemize}
\end{tcolorbox}

%%%% Consignes
\begin{tcolorbox}[colback=red!5!white,colframe=red!75!black,title= Consignes :]
\begin{itemize}
    \item Faire attention au matériel lors de son utilisation ;
\end{itemize}
\end{tcolorbox}

%%%% contexte
\section{Mise en évidence expérimentale}
\begin{tcolorbox}[colback=orange!5!white,colframe=orange!75!black,title= Expérience introductive :]
Réalisons l'expérience suivante : on éclaire par une lumière laser une cuve initialement vide. Puis on remplit la cuve d'un mélange d'eau et d'une substance fluorescente permettant de visualiser la lumière laser. Voici les schémas de l'expérience avant et après remplissage de la cuve :
\begin{center}
    \includegraphics[scale=0.4]{Images/TP/TP9/Experience_intro.png}
\end{center}
\textbf{\underline{Observations :}}
\texteTrouMultiLignes{~}{4}

\problematique{Peut-on relier les rayons refracté et réfléchi au rayon incident ? Peut-on mesurer un indice de réfraction d'un milieu ?}
\end{tcolorbox}


\begin{mdframed}[style=autreexo]
\textbf{\bsc{Liste du matériel}}
\begin{itemize}
    \item Une source de lumière collimatée avec son alimentation électrique ;
    \item Un demi-cylindre de plexiglas sur un disque gradué pivotant ;
    \item un ordinateur muni du logiciel tableau-grapheur ;
\end{itemize}
\end{mdframed}

%%%% documents

%%%%
\section{Lois de Snell-Descartes}
\begin{doc}{Quelques mots de vocabulaire}
\begin{center}
\vspace{-0.5cm}
     \includegraphics[scale=0.5]{Images/TP/TP9/Figure_propagation.PNG}
\end{center}
\importantbox{Les angles d'incidence $i_1$, de reflexion $r$ et de réfraction $i_2$ sont toujours comptés à partir de la normale à la surface de séparation.}
\end{doc}

\subsection{Loi de Snell-Descartes pour la réflexion}
\begin{multicols}{2}
\question{Allumer la lampe et régler le zéro du disque gradué à l'aide des vis de réglage.}{~}{0}
%\\
\question{Pour chaque valeur d'angle d'incidence $i_1$, mesurer l'angle de réflexion $r$ et compléter le tableau suivant :}{~}{0}
\begin{center}
    \includegraphics[width=0.4\textwidth]{Images/TP/TP9/Schema_exp.PNG}
\end{center}
\end{multicols}

\begin{center}
    \begin{tabular}{|c|C{0.06}|C{0.06}|C{0.06}|C{0.06}|C{0.06}|C{0.06}|C{0.06}|C{0.06}|C{0.06}|}
        \hline
        $i_1$ (en $\degree$) & 0 & 5 & 10 & 15 & 20 & 30 & 35 & 40 & 45 \\
        \hline
        $r$ (en $\degree$) & & & & & & & & &\\
        \hline
    \end{tabular}
\end{center}
%\\
\begin{tcolorbox}[colback=red!5!white,colframe=red!75!black,title=\textbf{Loi de Snell-Descartes pour la réflexion : }]
\texteTrouMultiLignes{~}{2}
\end{tcolorbox}

\subsection{Loi de Snell-Descartes pour la réfraction}
\begin{center}
    \includegraphics[scale=0.5]{Images/TP/TP9/DescartesvsKepler.png}
\end{center}

\problematique{Lequel de ces deux scientifiques avait raison ?}
\subsubsection{Expérience}

\begin{multicols}{2}
\question{Allumer la lampe et régler le zéro du disque gradué sur le chemin du faisceau lumineux.}{Réalisé en classe.}{0}
%\\
\question{Pour chaque valeur d'angle d'incidence $i_1$, mesurer l'angle de réfraction $i_2$ et compléter le tableau suivant (on ne prendra que 2 chiffres après la virgule pour le sinus) :}{~}{0}
\begin{center}
    \includegraphics[width=0.4\textwidth]{Images/TP/TP9/Schema_exp2.PNG}
\end{center}
\end{multicols}


\begin{center}
    \begin{tabular}{|C{0.1}|C{0.04}|C{0.04}|C{0.04}|C{0.04}|C{0.04}|C{0.04}|C{0.04}|C{0.04}|C{0.04}|C{0.04}|C{0.04}|C{0.04}|C{0.04}|C{0.04}|}
    \hline
        Angle $i_1$ (en $\degree$) & 0 & 5 & 10 & 15 & 20 & 25 & 30 & 35 & 40 & 45 & 50 & 60 & 70 & 80 \\
        \hline
        Angle $i_2$ (en $\degree$) &  &  &  &  &  &  &  &  &  &  &  &  &  &    \\
        \hline
        $\sin\left(i_1\right)$ &  &  &  &  &  &  &  &  &  &  &  &  &  &   \\
        \hline
        $\sin\left(i_2\right)$ &  &  &  &  &  &  &  &  &  &  &  &  &  &   \\
        \hline
        $\frac{i_1}{i_2}$ &  &  &  &  &  &  &  &  &  &  &  &  &  &  \\
        \hline
       $\frac{\sin\left(i_1\right)}{\sin\left(i_2\right)}$ &  &  &  &  &  &  &  &  &  &  &  &  &  &   \\
        \hline
    \end{tabular}
\end{center}

\subsubsection{Exploitation des résultats}
\question{\`{A} partir des résultats expérimentaux, tracer sur le papier millimétré $i_2$ en fonction de $i_1$.}{~}{0}
%\\
\question{Johannes Kepler avait-il raison ?}{Oui à faible angle d'incidence, les angles $i_1$ et $i_2$ sont proportionnels.}{0}
%\\
\question{Aller chercher le fichier \og TP9\_FeuilleCalculRefraction\_NomClasse\fg~sur l'ENT, dans l'espace documentaire de la classe.}{~}{0}
%\\
\question{Reporter les valeurs de $i_1$ et de $i_2$ dans le tableau Excel. Convertir les angles $i_1$ et $i_2$ en radian (rad). Puis calculer les sinus de ces angles en utilisant la fonction SINUS() du tableur Excel.}{~}{0}
%\\
\question{Tracer le graphique $\sin(i_2)=f(\sin(i_1))$ puis imprimez-le et coller-le sur votre compte-rendu.}{~}{0}
%\\
\question{En déduire qui de Kepler ou Descartes a énoncé la loi la plus valable. Justifier la réponse.}{~}{0}
%\\
On suppose que $\frac{\sin\left(i_2\right)}{\sin\left(i_1\right)} = \frac{n_1}{n_2}$ où $n_1$ et $n_2$ sont des nombres sans unités appelés \textcolor{red}{indices de réfraction}.\\
\question{En utilisant les fonctionnalités d’Excel (voir fiche méthode), déterminer une valeur expérimentale de $\frac{n_1}{n_2}$.}{~}{0}

\begin{tcolorbox}[colback=red!5!white,colframe=red!75!black,title=\textbf{Loi de Snell-Descartes pour la réfraction : }]
\texteTrouMultiLignes{~}{2}
\end{tcolorbox}
\question{En sachant que $n_1=n_{air}=1$, déterminer la valeur de $n_2=n_{plexiglas}$.}{~}{0}



\newpage
\papiermillimetre