%\newpage
%$ $
%\newpage

\renewcommand{\thesubsection}{\textcolor{red}{\Roman{section}.\arabic{subsection}}}
\renewcommand{\thesubsubsection}{\textcolor{red}{\Roman{section}.\arabic{subsection}.\alph{subsubsection}}}

\setcounter{section}{0}
\sndEnTeteTPO

\begin{center}
\begin{mdframed}[style=titr, leftmargin=60pt, rightmargin=60pt, innertopmargin=7pt, innerbottommargin=7pt, innerrightmargin=8pt, innerleftmargin=8pt]

\begin{center}
\large{\textbf{TP 0: Des mesures en tout sens}}
\end{center}

\end{mdframed}
\end{center}



\begin{tcolorbox}[colback=blue!5!white,colframe=blue!75!black,title=Objectifs de la séance :]
\begin{itemize}
    \item S'approprier du matériel de mesure
    \item Appréhender les incertitudes d'une mesure
\end{itemize}
\end{tcolorbox}

Pour cette première séance de TP de l'année, nous allons manipuler quelques outils de mesures pour comprendre les sources d'incertitudes associées à ces mesures.

\section{Mesure du temps}
\begin{mdframed}[style=autreexo]
\textbf{\bsc{Liste du matériel}}
\begin{itemize}
    \item un pendule pesant
    \item un chronomètre
\end{itemize}
\end{mdframed}
\begin{Large}{\textbf{Q :}} \end{Large} La période d'oscillation d'un pendule est définie comme le temps au bout duquel le pendule a fait un aller-retour. Selon vous, comment peut-on mesurer le plus précisément la période d'oscillation du pendule ? \\
\newline
\newline
\underline{$1^{\text{ère}}$ méthode :} Lâcher le pendule d'une faible hauteur. Chaque élève mesure une période d'oscillation. \textbf{Noter le protocole expérimental que vous avez utilisé. Comparer vos mesures entre vous. Proposer une/des explications sur l'écart entre vos différentes mesures.}:
\vspace{8cm}

\underline{$2^{\text{ème}}$ méthode :} Chaque binôme mesure 5 fois une période de manière indépendante. Notez les valeurs obtenues par binôme dans le tableau suivant :
\\

\begin{tabular}{|l|C{0.14}|C{0.14}|C{0.14}|C{0.14}|C{0.14}|C{0.14}|C{0.14}|C{0.14}|C{0.14}|C{0.14}|}
\hline
     Mesure n$^{\circ}$ & 1 & 2 & 3 & 4 & 5 \\
     \hline
     Période (en s) & & & & & \\
     \hline
\end{tabular}
\newline
\newline

\begin{tcolorbox}[colback=blue!5!white,colframe=white!75!black,title=Document 1 :]
La moyenne d'une mesure est la somme de toutes les valeurs mesurées divisée par le nombre de mesures effectuées :
\begin{equation*}
    \text{Moyenne} = \frac{\text{Mesure n$^{\circ}$1}+\text{Mesure n$^{\circ}$2} + ... + \text{Mesure n$^{\circ}$N}}{\text{N}}
\end{equation*}
L'écart-type représente la dispersion (c'est-à-dire à quel point les mesures changent d'une mesure à une autre) associée à une série d'une mesure. Elle est donnée par :
\begin{equation*}
    \text{Ecart-type }= \sqrt{\frac{\left(\text{Mesure n$^{\circ}$1}-\text{Moyenne}\right)^2+...+\left(\text{Mesure n$^{\circ}$N}-\text{Moyenne}\right)^2}{\text{N}}}
\end{equation*}
\end{tcolorbox}

\begin{tcolorbox}[colback=blue!5!white,colframe=white!75!black,title=Document 2 :]
On représente le résultat d'une mesure selon la manière suivante :
\begin{equation*}
    \text{Résultat} = \text{Moyenne} \pm \text{incertitude-type}
\end{equation*}
\end{tcolorbox}
\textbf{Travail à faire :} Calculer la moyenne notée $<T>$ de la période d'oscillation du pendule et évaluer qualitativement son incertitude-type. Comparer avec les valeurs obtenues avec la première méthode.\\
\underline{Pour les plus rapides :} calculer l'écart-type de vos mesures.

\newpage

\section{Mesure de la masse d'une bille de plomb}

\begin{mdframed}[style=autreexo]
\textbf{\bsc{Liste du matériel}}
\begin{itemize}
    \item 1 pied à coulisse
    \item 1 règle
    \item 2 balances
    \item des billes de plomb de différents diamètres
\end{itemize}
\end{mdframed}

\begin{tcolorbox}[colback=blue!5!white,colframe=white!75!black,title=Document 3 :]La masse $m$ d'une bille de plomb est reliée à son rayon $R$ et à sa masse volumique $\rho$ par la formule suivante :
\begin{equation*}
    m = \frac{4}{3}\pi R^3\times\rho
\end{equation*}
\end{tcolorbox}

\begin{Large}{\textbf{Q :}} \end{Large} Choisissez une bille de plomb, comment mesurer sa masse ? On donne $\rho=7,9$~g.cm$^{-3}$
\newline
\newline

\underline{$1^{\text{ère}}$ méthode :} Mesurer la masse sur les deux balances. \textbf{Noter votre résultat. \'{A} votre avis, quel est son incertitude ? Quel est la différence entre les deux balances ?}
\vspace{5cm}

\underline{$1^{\text{ère}}$ méthode :} Mesurer le rayon de la bille. En déduire la masse à l'aide du Document 3. \textbf{Comparer avec le résulat de la 1ère méthode.}


\newpage

\section*{Conclusion du TP}

\begin{tcolorbox}[colback=red!5!white,colframe=red!75!black,title=\textbf{On retiendra : }]
\begin{enumerate}
    \item L'instrument de mesure et le protocole expérimental influent sur le résultat d'une mesure,
    \item On écrit un résultat expérimental final en prenant en compte son incertitude :
    \begin{equation*}
        \text{Résultat} = \text{Moyenne} \pm \text{incertitude-type}
    \end{equation*}
    \item On peut améliorer la mesure d'une grandeur physique en répétant plusieurs fois le même protocole et en calculant la moyenne des mesures obtenues,
    \item On écrit un résultat en adaptant le nombre de chiffres significatifs.
\end{enumerate}
\end{tcolorbox}
