\renewcommand{\thesubsection}{\textcolor{red}{\Roman{section}.\arabic{subsection}}}
\renewcommand{\thesubsubsection}{\textcolor{red}{\Roman{section}.\arabic{subsection}.\alph{subsubsection}}}

\setcounter{section}{0}
\setcounter{document}{0}
\sndEnTeteActUn

\begin{center}
\begin{mdframed}[style=titr, leftmargin=60pt, rightmargin=60pt, innertopmargin=7pt, innerbottommargin=7pt, innerrightmargin=8pt, innerleftmargin=8pt]

\begin{center}
\large{\textbf{Activité 1 : Composition d'un mélange eau-huile - \textbf{Correction}}}
\end{center}

\end{mdframed}
\end{center}
\begin{Large}{\textbf{Q : Quelle est la proportion volumique et massique de l'huile et de l'eau présent dans l'éprouvette graduée sur le bureau du professeur ?}} \end{Large}
\newline

\`{A} la lecture sur l'éprouvette graduée, on lisait :
\begin{enumerate}
    \item $V_{huile}=15$~mL,
    \item $V_{eau}=30$~mL,
    \item $V_{Tot}= 45$~mL $=V_{huile}+V_{eau}$
\end{enumerate}

\section{Calcul de la proportion volumique de l'huile et de l'eau}
On en déduit directement la proportion volumique pour chaque espèce :
\begin{enumerate}
    \item $x_V(huile)=\frac{V_{huile}}{V_{Tot}}=\frac{15}{45}=33\%$
    \item $x_V(eau)=\frac{V_{eau}}{V_{Tot}}=\frac{30}{45}=67\%$
    \item \textcolor{red}{On vérifie bien que $x_V(huile)+x_V(eau)=100\%$}
\end{enumerate}

\section{Calcul de la proportion massique de l'huile et de l'eau}
Maintenant pour calculer la proportion massique de chaque espèce chimique, on doit d'abord calculer la masse de chaque espèce chimique. \\

\textbf{\underline{D'après le cours}}, la relation qui lie la masse et le volume est celle de la masse volumique à savoir :
\begin{equation*}
    \rho_{\text{espèce}} = \frac{m_{\text{(espèce)}}}{V_{\text{espèce}}}
\end{equation*}

On en déduit donc :
\begin{enumerate}
    \item $m_{\text{eau}}=\rho_{\text{eau}}\times V_{\text{eau}}=1,0~\text{g/mL}\times 30~\text{mL}=30$~g. \textcolor{red}{On ne garde que deux chiffres significatifs au résultat},
    \item $m_{\text{huile}}=\rho_{\text{huile}}\times V_{\text{eau}}=0,92~\text{g/mL}\times 15~\text{mL}=14$~g. \textcolor{red}{On ne garde que deux chiffres significatifs au résultat}
    \item $m_{Tot}=m_{\text{huile}}+m_{\text{eau}}=44$~g.
\end{enumerate}

On en déduit donc d'après le Document 2 la proportion massique des deux espèces :
\begin{align*}
    x_m(eau)&=\frac{m_{\text{eau}}}{m_{Tot}}=\frac{30}{44}=68\% & x_m(huile)&=\frac{m_{\text{eau}}}{m_{Tot}}=\frac{14}{44}=32\% 
\end{align*}
\textcolor{red}{On vérifie bien que $x_m(huile)+x_m(eau)=100\%$}