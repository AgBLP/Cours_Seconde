\newpage

\renewcommand{\thesubsection}{\textcolor{red}{\Roman{section}.\arabic{subsection}}}
\renewcommand{\thesubsubsection}{\textcolor{red}{\Roman{section}.\arabic{subsection}.\alph{subsubsection}}}

\setcounter{section}{0}
\sndEnTeteUn

\begin{center}
\begin{mdframed}[style=titr, leftmargin=60pt, rightmargin=60pt, innertopmargin=7pt, innerbottommargin=7pt, innerrightmargin=8pt, innerleftmargin=8pt]

\begin{center}
\large{\textbf{Chapitre 1 : Outils pour la Physique et la Chimie}}
\end{center}

\end{mdframed}
\end{center}
Ce chapitre présente quelques rappels sur les outils mathématiques, les méthodes et les représentations que nous utiliserons toute l'année en physique comme en chimie. Il est donc essetiel de maîtriser ces notions pour démarrer l'année sur d'excellentes bases. N'hésitez pas à le reconsulter toute l'année au besoin !

\begin{tcolorbox}[colback=blue!5!white,colframe=blue!75!black,title=Mots clés du chapitre :]
Puissance de $10$, unités SI, règles de sécurité en chimie
\end{tcolorbox}

\section{Puissance de 10}
\subsection{Représentation}
Les nombres très grands ou très petits s'écrivent à l'aide des puissances de 10 :
\begin{align*}
    10^n &= \underbrace{10\times10\times ... \times 10}_{\text{\textbf{n} "10 $\times$"}} = 1\underbrace{00...0}_{\text{\textbf{n} zéros}} \\
    10^{-n} &= \frac{1}{10^n} = 0,0....0\underbrace{1}_{\text{en \textbf{n-ième} position}}
\end{align*}

\underline{Exemples :} $10^9$ = \gap{...............................},    $10^{-5}$=\gap{......................}\\

\subsection{Opérations}
\begin{tcolorbox}[colback=red!5!white,colframe=red!75!black,title=\textbf{Règles de calculs des puissances de 10}]
Soit $a$ et $b$ deux nombres réels.
\begin{align*}
    10^{a}\times 10^b &= 10^{a+b} \\
    \frac{10^{a}}{10^b} &= 10^{a-b} \\
\end{align*}

\end{tcolorbox}
\underline{Effectuer les calculs suivants :}
\begin{align*}
    10^{30} \times 10^{50} &= \text{\gap{...............}} & \frac{10^{2033}}{10^{10}} &= \text{\gap{...............}} & \frac{10^2}{10^8}\times 10^5 &= \text{\gap{...............}}
\end{align*}
\subsection{Notation scientifique}

\begin{tcolorbox}[colback=green!5!white,colframe=green!75!black,title=\textbf{Ecriture scientifique d'un nombre}, upperbox=invisible]
En écriture scientifique, une valeur numérique s'exprime sous la forme :
\begin{equation*}
    a \times 10^n
\end{equation*}
avec $n$ un nombre entier relatif et $a$ un nombre décimal tel que : $1< a <10$.\\
\\
\end{tcolorbox}
\underline{Exemples :}
\begin{itemize}
    \item la masse $M_{L}$ de la Lune vaut $M_{Lune}=734$ $800$ $000$ $000$ $000$ $000$ $000$~kg et s'écrit donc plus simplement : $M_L=$\gap{....................}~kg
    \item La taille $d$ d'une molécule d'eau est de  $d=0,000$ $000$ $003$ $4$~m, soit en écriture scientifique : $d=$\gap{......................}~m
\end{itemize}

\subsection{Ordre de grandeur}
\begin{tcolorbox}
[colback=green!5!white,colframe=green!75!black,title=\textbf{Ordre de grandeur d'un nombre}]
L'ordre de grandeur d'un nombre et la puissance de 10 la plus proche de ce nombre.
\end{tcolorbox}
\underline{Exemples :}
\begin{itemize}
    \item Ordre de grandeur de 30 000 = \gap{...........}
    \item Ordre de grandeur de 0,000 007 = \gap{...........}
    \item Ordre de grandeur de 8 700 000 = \gap{...........}
    \item Ordre de grandeur de 0,000 256 = \gap{...........}
\end{itemize}

\section{Unités}
\subsection{Les unités du système international SI}
Depuis 2019, la communauté scientifique internationale, à travers la "Conférence générale des poids et mesures", a adopté un système d'unités universel. Leur valeur sont définies à partir de 7 constantes universelles dont les valeurs exactes ont été définitivement "fixées" en Novembre 2018 (\textit{source Wikipédia}). 
\begin{figure}[!h]
    \centering\includegraphics[scale=0.4]{Cours/SI_Logo_with_defining_constants.png}
    \caption{Logo du système SI. }
    \label{fig:Syteme_SI}
\end{figure}
\begin{table}[!h]
    \centering
    \begin{tabularx}{\textwidth}{| X | X | c | X |}  \hline
Grandeur physique & Unité SI & Symbole de l'unité & Appareil de mesure \\
\hline
Masse & kilogramme & kg & balance \\
Temps & seconde & s & Chronomètre \\
Longueur & mètre & m & règle \\
Température & kelvin & K & thermomètre \\
Intensité électrique & ampère & A & ampèremètre \\
Quantité de matière & mole & mol & balance\\
Intensité lumineuse & candela & cd & Luxmètre \\
\hline
\end{tabularx}
    \caption{Tableau des unités du SI et}
    \label{tab:SI}
\end{table}
\importantbox{Nous utilisons courament d'autres unités qui nous paraissent bien plus pratiques dans leur utilisation. Par exemple, nous utilisons plus facilement les degrés celsius $^\circ$C pour la température ou encore les tailles XS, S, M, L, ... pour les vêtements (même si la relation entre ces dernières \og unités \fg et les unités du SI semble très obscure...).}


\subsection{Préfixes des unités.}
Il est très souvent utile et agréable de remplacer l’écriture scientifique d’un nombre par \textbf{un nombre écrit sans puissance de 10 suivi d'un multiple ou sous-multiple de l'unité du nombre}.
\begin{table}[!h]
    \centering
    \begin{tabularx}{\linewidth}{|c|c|c|c|}
    \hline
    Multiple ou sous-multiple & Facteur par lequel l'unité est multipliée & Préfixe & Symbole \\ 
    \hline 
    Multiple & 1 000 000 000 000 = $10^{12}$ & téra & T \\
    \hline
    Multiple & 1 000 000 000 = $10^{9}$ &  & \\
    \hline
    Multiple & 1 000 000 = $10^{6}$ &  &  \\
    \hline
    Multiple & 1 000 = $10^{3}$ &  & \\
    \hline
    Multiple & 100 = $10^{2}$ &  &  \\
    \hline
    Multiple & 10 = $10^{1}$ &  &  \\
    \hline
    Sous-Multiple & 0,1 = $10^{-1}$ &  & \\
    \hline
    Sous-Multiple & 0, 01 = $10^{-2}$ &  & \\
    \hline
    Sous-Multiple & 0, 001 = $10^{-3}$ &  & \\
    \hline
    Sous-Multiple & 0, 000 001 = $10^{-6}$ & &  \\
    \hline
    Sous-Multiple & 0, 000 000 001 = $10^{-9}$ &  & \\
    \hline
    Sous-Multiple & 0, 000 000 000 001 = $10^{-12}$ &  & \\
    \hline
    Sous-Multiple & 0, 000 000 000 000 001 = $10^{-15}$ &  & \\
    \hline
    \end{tabularx}
    
    \caption{Tableaux des préfixes des multiples et sous multiples}
    \label{tab:chap1_multiples}
\end{table}

\section{Règles de sécurité en TP de chimie}
\subsection{Les règles de sécurité}
\begin{tcolorbox}[colback=red!5!white,colframe=red!75!black,title=\textbf{En TP de chimie :}]
\vspace{10cm}
\end{tcolorbox}

\subsection{Les pictogrammes de sécurité et EPI}
\begin{figure}[!h]
    \centering
    \includegraphics[scale=1]{Cours/Pictogrammes.png}
    \caption{Pictogrammes de sécurité}
    \label{fig:enter-label}
\end{figure}

\begin{figure}[!h]
    \centering
    \includegraphics[scale=0.9]{Cours/picto_gant_blouse_lunette.jpg}
    \caption{Logo des \'{E}quipements de Protection Individuelle (EPI)}
    \label{fig:enter-label}
\end{figure}

\importantbox{Si vous avez un doute concernant un produit non étiquetté, il ne faut pas hésiter à regarder sa fiche toxicologique sur le site de l'INRS (Institut National de Recherche et de Sécurité) : \url{https://www.inrs.fr/publications/bdd/fichetox.html}}

\underline{\textbf{Exemple :}}
\begin{figure}[!h]
    \centering
    \includegraphics[scale=0.7]{Cours/Exemple_fiche_toxico.png}
    \caption{Exemple d'une fiche toxicologique donnée par le site de l'INRS.}
    \label{fig:enter-label}
\end{figure}