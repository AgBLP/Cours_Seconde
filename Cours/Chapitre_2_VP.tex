\renewcommand{\thesubsection}{\textcolor{red}{\Roman{section}.\arabic{subsection}}}
\renewcommand{\thesubsubsection}{\textcolor{red}{\Roman{section}.\arabic{subsection}.\alph{subsubsection}}}

\setcounter{section}{0}
\sndEnTeteCoursDeux

\begin{mdframed}[style=titr, leftmargin=60pt, rightmargin=60pt, innertopmargin=7pt, innerbottommargin=7pt, innerrightmargin=8pt, innerleftmargin=8pt]

\begin{center}
\large{\textbf{Chapitre 2 : Les solutions acqueuses}}
\end{center}
\end{mdframed}

\subsubsection{La solubilité}
\begin{tcolorbox}[colback=green!5!white,colframe=green!75!black,title=\textbf{Définition de la solubilité}]
La solubilité, notée $s$, d'un corps pur dans un solvant est la masse (ou quantité de matière) maximale de ce corps qu'on peut dissoudre dans $1$~L de solvant. Elle s'exprime en $\mathbf{g.L^{-1}}$.\\
Il s'agit d'une \textbf{concentration massique} !
\end{tcolorbox}
