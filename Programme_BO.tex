\begin{center}
\begin{Large}
\textbf{Programme 2023-2024 de la classe de seconde}
\end{Large}
\end{center}

\begin{table}[!h]
\centering
\setcellgapes{5cm}
\begin{tabularx}{\textwidth}{|X|X|X|}

\rowcolor{black}\multicolumn{3}{|p{\textwidth-0.5cm}|}{\centering\large\textbf{\textcolor{white}{Thème 1 : Constitution et transformations de la matière}}} \\
\hline
Chapitres & Notions abordées & Idées TP\\
\hline
Corps purs et mélanges au quotidien & \begin{itemize} 
\item Mélanges homogènes et hétérogènes,
\item corps purs
\item identification d'espèces chimiques (ex : T$_{eb}$) 
\item compo massique mélange\end{itemize} & \begin{itemize} 
\item CCM 
\item Tests chimiques pour révéler H$_2$, CO$_2$, O$_2$, H$_2$O,
\item dosage d'une espèce dans un mélange 
\end{itemize}\\
\hline 
Solution acqueuse, un exemple de mélange & \begin{itemize}
\item solvant, soluté
\item concentration massique, solubilité
\item dosage par étalonnage
\end{itemize} & \\
 \hline
Modélisation de la matière à l'échelle macroscopique & & \\
 \hline
Du macro au micro, de l'espèce chimique à l'entité & & \\
 \hline
\end{tabularx}
\end{table}

\begin{table}[!h]
\centering
\setcellgapes{5cm}
\begin{tabularx}{\textwidth}{|X|X|X|}

\rowcolor{black}\multicolumn{3}{|p{\textwidth-1cm}|}{\centering\large\textbf{\textcolor{white}{Thème 1 bis : "L’énergie : conversions et transferts"}}} \\
\hline
Chapitres & Notions abordées & Idées TP\\
\hline
Corps purs et mélanges au quotidien & \begin{itemize} 
\item Mélanges homogènes et hétérogènes,
\item corps purs
\item identification d'espèces chimiques (ex : T$_{eb}$) \end{itemize} & Dosage sucre \\
\hline 
Solution acqueuse, un exemple de mélange & & \\
 \hline
Modélisation de la matière à l'échelle macroscopique & & \\
 \hline
Du macro au micro, de l'espèce chimique à l'entité & & \\
 \hline
\end{tabularx}
\end{table}

%Programme de Seconde 2023 : 

%
%	- 
%	- Mouvement et interactions
%	- Ondes et signaux



%Dans Constitution et transformations de la matière :
%	- Description et caractérisation de la matière à l’échelle macroscopique à savoir :
%		- 
%		- 
%	- Modélisation de la matière à l'échelle macroscopique
%		- Du macro au micro, de l'espèce chimique à l'entité
%		- Noyau de l'atome
%		- Cortège électronique
%		- Vers des entités plus stable chimiquement
%		- Compter les échantillons dans la matière
%	- Modélisation des transformations de la matière et transfert d’énergie
%		- Transformation physique 
%		- Transformation chimique
%		- Transformation nucléaire

